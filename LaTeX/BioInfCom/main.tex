% -------------------- 导言 -------------------- %
\documentclass{ctexart}
\usepackage{mhchem}
% 设置页边距
\usepackage{geometry}
\geometry{left=3.18cm, right=3.18cm, top=2.54cm, bottom=2.54cm}

% 页眉
\usepackage{fancyhdr}
\pagestyle{fancy}
\fancyhead[L]{\small{\CJKfontspec{SimHei}第九届生物信息设计与技能竞赛}}
\fancyhead[R]{\small{\CJKfontspec{SimHei}大豆根系微生物的基因鉴定与基因功能预测}}
\renewcommand{\headrulewidth}{1pt} % 分隔线宽度 1 磅

% 字体
\usepackage{fontspec}
\setmainfont{Times New Roman} % 英文正文字体
\setCJKmainfont{SimSun} % 中文正文字体
\setCJKsansfont{SimHei}

% 设置标号深度
\setcounter{secnumdepth}{3}
\author{}
\date{}

\usepackage{caption}
\captionsetup[table]{
    font={small,bf },
    labelsep=quad,
    skip=0pt
}

% 表格
\usepackage{booktabs}

\title{\vspace*{-1.5cm} \CJKfontspec{SimHei}大豆根系微生物的基因鉴定与基因功能预测}


% \ctexset{
%     % 修改 section。
%     section={   
%         format=\heiti\raggedright\zihao{4} % 设置 section 标题为黑体、右对齐、小4号字
%     },
%     % 修改 subsection。
%     subsection={   
%         format=\heiti\zihao{-4}
%     },

%     subsubsection={
%         format=\songti\zihao{-4}
%     }
% }

% -------------------- 正文 -------------------- %
\begin{document}

    \maketitle\thispagestyle{fancy}
    \vspace*{-1.5cm}

    \raggedleft
    {\zihao{5} \heiti  ——竞赛单元“2-1基因组”}

    ~\\

    \raggedleft
        团队名称:水哥微生物小队\\
        指导老师:郑金水\\
        团队成员:张敦彪,张子栋,颜旭,姚代洪\\
    
    ~\\

    \raggedright
    {\zihao{4} \heiti 摘要}

    {\zihao{5} \qquad 高通量培养与鉴定可用于从给定环境和植物物种的本地根微生物组样本中建立一个分类学上全面的细菌培养收集,通过 16S rRNA基因扩增子测序对根样品的细菌多样性进行评价,筛选出具有相应16S rRNA 基因序列的培养菌。通过对大豆根系微生物的高通量培养筛选可以鉴定其基因的类型,进而对其基因功能进行预测。}
    
    ~\\

    {\heiti \zihao{-4} \raggedright 关键字:} {\zihao{5} 高通量培养,基因鉴定,功能预测}

    
    % % 目录设为页码 0(从正文起为第一页)
    % \setcounter{page}{0}

    % % 目录的页脚设为空
    % \thispagestyle{empty}


    % \chapter{Hello}

    \section{引言}

    \subsection{根系微生物群}

    \qquad 根系微生物是指生活在植物根系附近的微生物群落,包括细菌、真菌和古菌等。它们与植物根系形成一种共生关系,相互作用并互利共生。根系微生物可以对植物生长和健康产生积极影响。研究根系微生物对于促进植物生长健康、提高作物产量质量、维护土壤生态系统健康以及发掘新的生物资源具有重要意义。


    \subsection{高通量培养}

    \qquad 高通量培养是一种用于微生物的培养和筛选的技术,通过该技术可以大大提高微生物的培养效率和筛选速度,同时也有助于扩展微生物资源库,挖掘未知微生物代谢产物等领域的研究。
    
    \qquad 传统的微生物培养方法是基于单个菌落的分离和培养,但是由于绝大部分微生物无法在常规培养条件下生长,很多微生物资源依然无法被发现和利用。而高通量培养可以利用微型板或微流控芯片等设备将微生物单个分离后进行多重并行培养,以大幅提高微生物的培养效率。此外,高通量培养还可以借助自动化解决方案,实现对大量微生物的快速筛选。
    

    \section{材料与方法}

    \subsection{培养基的制作}

        实验中需要三种不同的培养基(三种培养基均高压蒸汽灭菌):

    \begin{table}[htb]
        \caption{实验所需培养基}
        \centering
        \begin{tabular}{cc}
            \toprule
            培养基类型 & 配置方法\\
            \midrule
            TSB 培养基 & $\mathrm{30~g/L}$ TSB,蒸馏水定容\\
            10\% TSB培养基 & TSB 培养基按照体积分数 10\% 进行稀释\\
            10\% TSB培养基+丙酮酸钠 & $\mathrm{30~g/L}$ TSB,$\mathrm{10~mM/L} $丙酮酸钠,蒸馏水定容\\
            \bottomrule
        \end{tabular}
    \end{table}

    \subsection{大豆根系菌群的获取}

    \qquad 在华中农业大学 BMB 课题组大豆孢囊线虫大棚中摘取自然条件下生长、表型正常的未开花的大豆,在取样前不宜过多浇水,以免造成大豆根系附着较多土壤,不易除去。用手轻轻拨去大豆根系的土壤以去除大量土壤团聚体,此后将大豆根系置于手掌虎口处轻轻抖动,去除碎土。

    \subsubsection{根际样本}
        \qquad 超净工作台中将大豆根系置于无菌滤纸上,使用无菌刀片截取其下方约3cm的根系,总共截取3g的大豆根系样本。拿取 $\mathrm{10x~PBS}$ 溶液,按照体积分数的 10\% 稀释为 $\mathrm{1x~ PBS}$ 溶液约 100~mL 置于两个 50~mL 的离心管中。将截取的根系样本置于 50~mL 的 $\mathrm{1x~ PBS}$ 溶液中来回水平摇匀,将其以 1800~rpm 的速度震荡 20~min,取出其中的根系,将溶液过500目筛,获得根际样本。

    \subsubsection{根内样本}
        \qquad 按照上述方法新截取约 3~g 的大豆根系,使用无菌水持续冲洗,直至除去所有的土壤颗粒。将根系转移至一个新的含有 50~mL 的 1x~PBS 溶液的离心管中,以 1800~rpm 的速度震荡处理 20~min,从离心管中取出根系,使用无菌滤纸吸干,从根表面除去 PBS 溶液,使用无菌水冲洗三次,将根系置于含有 50~mL 无菌水的离心管中,进行超声处理 5~min,将根系取出置于含有 50~mL 的75\%体积分数的酒精溶液的离心管中,震荡 1~min,取出根系,使用无菌水冲洗三次,将根系悬浮在 50~mL 的 10~mMol/L 的无菌$\ce{MgCl2}$溶液中,使用无菌研钵研碎根系直至其均匀,将研钵内液体过 500 目筛获取根内样本。

        \qquad 根内样本与根际样本均置于 4℃ 储存。

    \subsection{分液与流式细胞分选}

    \qquad 用分液仪将液体培养基分装到 60 个 384 孔板,每孔 7~μL,再用流式细胞分选仪将根际与根内样本转移到每个孔中并做好标记。


    \subsection{PCR 扩增与电泳鉴定}
    \qquad 将得到的八连管每个孔中按比例加入1μl F引物、1μl R引物、7μl 无菌水、10μl mix与原1μl菌液形成20μl体系,按照对应程序(如表1)将八连管放入pcr仪器处理。
    \begin{table}[!htb]
        \centering
        \caption{PCR 反应体系}
        \begin{tabular}{cc}
            \toprule
            成分 & 体积(μL)\\
            \midrule
            F 引物 & 1\\
            R 引物 & 1\\
            \ce{ddH2O} & 7\\
            Mix & 10\\
            原样液 & 1\\
            \bottomrule
        \end{tabular}
    \end{table}


    \begin{table}[!htb]
        \centering
        \caption{PCR 程序设定}
        \begin{tabular}{cccc}
            \toprule
            循环数 & 流程 & 温度 & 时间\\
            \midrule
            1 & 预变性 & 98℃ & 10~min\\
            & 变性 & 95℃ & 30~s\\
            2-36 & 复性 & 55℃ & 30~s\\
            & 延伸 & 72℃ & 90~s\\
            37 & 终延伸 & 72℃ & 5~min\\
            \bottomrule
        \end{tabular}
    \end{table}


\end{document}