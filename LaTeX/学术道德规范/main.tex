\documentclass[UTF8]{ctexart}

\usepackage{fancyhdr}
\fancyhead[L]{\LaTeX 在论文排版中的应用}
\fancyhead[R]{生信2001 张子栋 2020317210101}

\title{\LaTeX 在论文排版中的应用}
\author{生信2001 张子栋 2020317210101}
\date{\today}

\begin{document}

\maketitle
\thispagestyle{empty}
\clearpage

\tableofcontents
\setcounter{page}{1}
\clearpage

\thispagestyle{empty}
\begin{abstract}
    \LaTeX 是一种高质量的排版系统,主要用于长篇技术或科技文档的编写。它可以让作者专注于内容而不是格式,\LaTeX 可以自动处理标题、章节、图表、参考文献等格式问题。
\end{abstract}
\clearpage

\pagestyle{fancy}
\setcounter{page}{1}

\section{论文排版系统概述}
\subsection{论文排版系统的分类}
在论文排版中,常用的有字处理软件字处理软件 MS Office Word \cite{word}和排版引擎 \LaTeX。Word 是一款功能强大且广泛使用的付费字处理软件,上手简单,学习成本低,并且有可视化界面,是一种所见即所得的文本编辑器。然而,与 \LaTeX 相比,Word 的排版能力较为有限,\LaTeX 是一种基于 \TeX \cite{tex} 的排版系统,被广泛用于学术界和科技界。它提供了强大的数学公式排版、参考文献管理和自动化排版等功能。与 Word 相比,\LaTeX 的排版质量更高,并且可以生成专业且精美的文档。

\subsection{论文排版系统的历史沿革}

最早的排版系统是在中国11世纪左右发明的活字印刷术,使用可移动的泥字。在1297年,王祯改进了这个系统,发明了一种旋转的字盘,提高了印刷速度。在1440年,约翰内斯·古腾堡在德国美因茨引入了现代印刷机,使用金属字和机械压力。这种方法产生了一种被许多作者和出版商欣赏的“经典风格”。在19世纪,发展出了各种排版机,如铅字机和单字机,它们自动化了铸造和排列金属字的过程。在20世纪,光敏排版取代了金属字,使用光敏纸或胶片,它们被计算机或其他设备生成的字符图像曝光。在1978年,Knuth, D. E. 创建了 \TeX,一种为高质量的数学和技术出版物设计的排版系统。\TeX 基于一组命令和宏,控制文本和符号的布局和外观。\TeX 仍然在学术界广泛使用。在1983年,Leslie Lamport 开发了\LaTeX,一种为\TeX 提供的宏包,简化了文档的格式化,并提供了许多功能,如交叉引用、参考文献、表格、图形等。\LaTeX 是学术写作中最流行的排版系统之一。同年,微软发布了Word,一种文字处理器,允许用户在图形用户界面上创建和编辑文档。Word支持许多排版功能,如字体、样式、边距、页眉、页脚等。Word 是个人和专业用途中最广泛使用的应用程序之一。

\section{论文排版系统用法概述}
\subsection{Word 用于理工科论文排版的不足}
    格式控制不精确。Word 的格式控制是基于段落和样式的,用户需要为每个段落设置相应的样式,以保证论文的格式统一。然而,Word 的样式有时会出现混乱或丢失的情况,导致论文的格式出现错误或不一致。此外,Word 的样式也不能完全覆盖所有的格式要求,例如页眉页脚、目录、图表、公式等,用户还需要进行额外的设置或调整。

    图表和公式处理不方便。Word 的图表和公式功能相对较弱,用户需要借助其他软件或插件来制作或编辑图表和公式。例如,Word 的图表功能不能支持复杂的数据分析和可视化,用户需要使用 Excel 或 SPSS 等软件来生成图表,然后再插入到 Word 中。Word 的公式功能也不能支持高级的数学符号和排版,用户需要使用 MathType 或 \LaTeX 等软件来编写公式,然后再复制到 Word 中。这些操作不仅增加了用户的工作量,也增加了论文的错误风险。

    文件管理和协作不便捷。Word 的文件管理功能较为简单,用户需要自己管理论文的各个部分和版本,以防止文件丢失或混乱。例如,用户需要为论文的每个章节创建单独的文件,然后再将它们合并为一个完整的文件。用户也需要为论文的每次修改创建备份文件,以便于恢复或比较。Word 的协作功能也较为有限,用户需要通过邮件或网络盘等方式来共享或交换文件,以实现论文的多人合作或审阅。这些方式不仅效率低下,也容易造成文件冲突或丢失。
\subsection{\LaTeX 在论文排版中的应用}

相比 Word,\LaTeX 有以下优点:

专业的排版输出能力。LaTeX 对包含章节、交叉引用、图表的长篇文档具有很强的格式控制能力。LaTeX 还具有优秀的数学公式、图形和参考文献处理能力,无出其右者。

内容与格式分离。LaTeX 的设计目标是让作者能够无需关注版式设计,只需专注于内容创作。LaTeX 为单个元素定义了样式,这些元素在整个文档中风格一致;用户不需要在每次添加元素时都进行编辑格式。因此,用户可以安心地写作,只需最终做一次所有的格式编辑。

强大的可扩展性。LaTeX 有数以千计的宏包用于补充和扩展 LaTeX 的功能。这些宏包可以在 CTAN(Comprehensive TeX Archive Network)上找到。用户可以根据自己的需要选择合适的宏包来实现所需的功能和效果。

平台独立和免费开源。LaTeX 和 TeX 及相关软件是跨平台、免费、开源的。无论用户使用的是 Windows,macOS,GNU/Linux 还是 FreeBSD 等操作系统,都能轻松获得和使用这一强大的排版工具,并且获得稳定的输出。

利于出版和协作。许多期刊和出版商明确要求或者推荐作者用 LaTeX 格式化他们的论文。例如,Nature,Elsevier,SAGE 等顶级期刊和出版商都提供了 LaTeX 的模板和选项。他们更喜欢作者用 LaTeX 提交,因为 LaTeX 更容易编辑和审阅。此外,LaTeX 的源文件是纯文本文件,可以方便地与合作伙伴或审稿人共享或交换,无需考虑软件环境和文件依赖的问题2。
% 公式排版
% 引用管理
% 版本控制

\section{结论与展望}
在许多提供 \LaTeX 模板的网站上,国外大学的毕业论文模板数量远多于国内大学,这与中文互联网的滞后性有关,因为在计算机技术引入中国并流行开始,微软的 Word 和求伯君的 WPS 已经发布,而在这两个软件发布前,\LaTeX 已经在国外有了长远的发展,所以在国内 \LaTeX 远不及 Word 流行。并不是说 \LaTeX 与 Word 相比哪个更优秀,而是 \LaTeX 生成的 PDF 文档在国内的接受程度没有 Word 文档高,这说明 \LaTeX 还需要更多的推广和应用,从个人来说,可以从制作论文模板开始。

此外 typst 也是近期发布的一种排版引擎,为了解决 \LaTeX 的痛点而产生,仅发布6天内便在 GitHub 上获得 10000 star,这说明 \LaTeX 虽然功能相对强大,但仍有可以进步的地方, typst 的迅速流行就是一个很好的说明,但是毕竟处于软件开发初期,部分功能语法还没有完全固定,也有许多功能不完善,特别是中文排版方面,但包括我在内,很多人看法 typst 的发展。

\begin{thebibliography}{99}
    \bibitem{tex} Knuth, D. E. (1979). TeX: A system for technical text. In The art of computer programming: Seminumerical algorithms (Vol. 2, pp. 471-505). Addison-Wesley.
    \bibitem{word} 微软公司. (2021). Microsoft Word. https://www.microsoft.com/zh-cn/microsoft-365/word 
    
\end{thebibliography}

\end{document}