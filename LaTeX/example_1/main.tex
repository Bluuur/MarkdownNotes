\documentclass{ctexart}
% 设置页边距
\usepackage{geometry}
\geometry{left=3.18cm,right=3.18cm,top=2.54cm,bottom=2.54cm}

% 设置目录显示的深度
\setcounter{tocdepth}{3}

% 设置标号深度
\setcounter{secnumdepth}{3}

% 图片
\usepackage{graphicx}
% 图片位置
\usepackage{float}
% 并排
\usepackage{subfigure}
\usepackage{parskip}

\usepackage{fancyhdr}
\pagestyle{fancy}
\fancyhead[L]{\small \CJKfontspec{SimHei}武汉新助研生物科技有限公司}          % 左页眉
\fancyhead[R]{\small \CJKfontspec{SimHei}16S扩增子测序数据分析报告}    % 右页眉
% \fancyhead[L]{中间页眉}
% \fancyfoot[L]{左页脚}
\fancyfoot[C]{第 \thepage 页 \quad 共 \pageref{LastPage} 页}
% \fancyfoot[R]{右页脚}
\renewcommand{\headrulewidth}{1pt}

\title{16s Report}
\author{author}
\date{\today}

% 正文区
\begin{document}


    \maketitle

    \newpage

    \tableofcontents
    
    % 目录设为页码 0(从正文起为第一页)
    \setcounter{page}{0}

    % 目录的页脚设为空
    \thispagestyle{empty}

    \newpage

    \section{16S扩增子测序数据分析方案流程}


    \section{样本与测序信息}
    \subsection{方法描述}
    \section{样本信息}
    \section{数据预处理}
    \section{物种注释}
    \section{多样性分析}
    \section{差异分析}
    \section{功能预测}



\end{document}