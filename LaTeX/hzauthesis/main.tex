\documentclass{ctexart}

\ctexset{
	section={
		%format用于设置章节标题全局格式,作用域为标题和编号
		%字号为三号,字体为黑体,左对齐
		%+号表示在原有格式下附加格式命令
		format+ = \zihao{3} \heiti \raggedright,
	},
	subsection={
		format+ = \zihao{-3} \heiti \raggedright,
	},
	subsubsection={
		format+ = \zihao{4} \heiti \raggedright,
	}
}


\usepackage{zhlipsum}

\usepackage[a4paper]{geometry}
% 设置页边距
\geometry{
    twoside,
    top=25mm, 
    bottom=25mm, 
    inner=32mm, 
    outer=25mm,
    headheight=3mm, headsep=4mm,
    footskip=5mm,
}


\usepackage{fancyhdr}
% 设置页眉
\pagestyle{fancy}
\fancyhead{}
    \fancyhead[CO]{论文题目} % 奇数页页眉
    \fancyhead[CE]{华中农业大学$\times\times$届学士学位(毕业论文)} % 偶数页页眉
\fancyfoot{}
    \fancyfoot[CO,CE]{\thepage}

\begin{document}
% 论文独创性声明及使用授权书

% 目录
\pagenumbering{Roman} %设置罗马数字页码
\begin{center}
    \tableofcontents    
\end{center}

\setcounter{page}{1}



\clearpage

\pagenumbering{roman} %设置罗马数字页码
\begin{abstract}
    \zhlipsum
\end{abstract}

\clearpage

% \begin{abstract}*
%     \zhlipsum
% \end{abstract}

\pagenumbering{arabic} %设置阿拉伯数字页码

\zhlipsum

\section{前言}
\section{材料与方法}
\section{结果与分析}
\section{讨论}

\subsection{二级标题}

\subsubsection{三级标题}

\end{document}