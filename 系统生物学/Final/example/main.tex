\documentclass{ctexart}

\usepackage{mhchem}
\usepackage{float}
\usepackage{booktabs}
% 设置页边距
\usepackage{geometry}
\geometry{left=3.18cm,right=3.18cm,top=2.54cm,bottom=2.54cm}

% 设置目录显示的深度
\setcounter{tocdepth}{3}

% 设置标号深度
\setcounter{secnumdepth}{3}

\title{对于生物体内ATP合成反应的系统生物学建模}
\author{生信2001张子栋}
\date{\today}

% 正文区
\begin{document}


\maketitle

\newpage

\tableofcontents

% 目录设为页码 0(从正文起为第一页)
\setcounter{page}{0}

% 目录的页脚设为空
\thispagestyle{empty}

\newpage

\section{生物学背景}
ATP合成是细胞代谢中一个非常重要的过程,它可以利用电子传递链产生的质子驱动力,将ADP和无机磷酸合成为ATP,从而提供细胞所需的能量。ATP合成反应是一个氧化磷酸化的过程,发生在真核细胞的线粒体内膜或原核生物的细胞质中;ATP合成反应是一个偶联反应,即有机物在体内氧化时释放的能量通过呼吸链供给ADP与无机磷酸合成ATP。氧化磷酸化过程中 ATP 的合成与电子传递链上的几种复合体密切相关。电子传递链(ETC)上有五种酶复合物支持 OXPHOS 系统运转:复合物 I (也称 CI 或 NADH:泛醌氧化还原酶),复合物 II (也称 CII 或琥珀酸脱氢酶 SDH),二聚体复合物 III2 (也称 CIII2 或细胞色素 bc1 氧化还原酶),复合物 IV (也称 CIV 或细胞色素 c 氧化酶)。由复合物 I-IV 生成的质子梯度随后被复合物 V,也就是 ATP 合酶所利用,将 ADP 磷酸化为ATP。本文从系统生物学的角度对ATP合成过程进行建模和求解,基于已知的实验数据和资料,建立一个可计算的数学模型,通过模拟和求解此模型,获得有关ATP合成过程的定量信息,以探究ATP合成的机制。

\section{模型的建立}
\subsection{ATP合成的相关反应}
ATP合成包括以下反应:
\quad\\

\ce{ADP + Pi ->[ATP Synthase] ATP}

\ce{ATP ->[ATP ase] ADP + Pi}

\ce{NADH + H ->[Complex I] FADH_2}

\ce{FADH_2 + H ->[Complex II] NADH}

\ce{NADH + H + \frac{1}{2} O_2 ->[Complex IV] NAD^+ + H2O}
\quad\\

其中,第一个反应是由复合物V(ATP合酶)催化的,产物是ATP,这个反应被称为磷酸化;第二个反应是由复合物V(ATP酶)催化的,产物是 ADP 和 Pi,这个反应被称为解磷酸化。第三个反应是由复合物I(NADH-辅酶Q氧化还原酶)和复合物III(细胞色素bc1复合体)催化的,产物是ATP,这个反应被称为氧化磷酸化。第三个是由复合物I催化,产物是$NAD^+$和$H^+$,第四个反应是由复合物II(脂肪酰辅酶Q还原酶)催化,产物是FAD和$H^+$。这两个反应都是电子传递链中的反应。

\subsubsection{各反应反应速率常数}
各反应反应速率常数如下表,$k_1 - k_6$ 分别对应五个反应,$K_m$ 是酶促反应的 Michaelis 常数。

\begin{table}[H]
    \centering
    \begin{tabular}{ccc}
        \toprule
        反应速率平衡常数 & 数值 & 单位\\
        \midrule
        $k_1$ & 200.0 & $mM^{-2} \cdot s^{-1}$\\
        $k_2$ & 10.0 & $s^{-1}$\\
        $k_3$ & 0.1 & $s^{-1}$\\
        $k_4$ & 0.01 & $s^{-1}$\\
        $k_5$ & 2.0 & $mM^{-1} \cdot s^{-1}$\\
        $k_6$ & 0.1 & $mM^{-1} \cdot s^{-1}$\\
        $K_M$ & 0.1 & $mM$ \\
        \bottomrule
    \end{tabular}
    \caption{\textbf{模型中各物质初始浓度}}
\end{table}

\subsection{建立常微分方程组}
\subsubsection{模型需要满足的假设}

采用常微分方程组的方法,建立 ATP 合成的机制动态数学模型。该模型包括以下假设:

\begin{enumerate}
    \item ATP 合成过程中,存在着多个反应步骤,其中一些反应需要能量输入,一些产生能量。
    \item ATP 合成过程中,各个分子组分之间及其反应关系是相互联系的。
    \item ATP 合成反应的速率与当时反应物的浓度有关,可以用速率方程来描述。
    \item ATP 合成反应的速率方程可以用微积分方法来求解,得到一个积分反应速率方程。
\end{enumerate}

\subsubsection{常微分方程组的建立}

基于 2.2.1 中的假设,建立一个包含五个动态变量(ATP,ADP,Pi,NADH,H)的微分方程组:

$$
    \begin{aligned}
        \frac{\mathrm{d}[\text{ATP}]}{d\mathrm{d}t} & = k_{1}[\text{ADP}] [\text{Pi}] - k_2 [\text{ATP}]                                                             \\
        \frac{\mathrm{d}[\text{ADP}]}{\mathrm{d}t}  & = k_{2} [\text{ATP}] - k_{1} [\text{ADP}] [\text{Pi}]                                                          \\
        \frac{\mathrm{d}[\text{Pi}]}{\mathrm{d}t}   & = k_{1} [\text{ADP}] [\text{Pi}] - k_{2} [\text{ATP}]                                                          \\
        \frac{\mathrm{d}[\text{NADH}]}{\mathrm{d}t} & = -k_{3} [\text{NADH}] [\text{H}] + k_{4} [\text{FADH}_2] [\text{H}]                                           \\
        \frac{\mathrm{d}[\text{H}]}{\mathrm{d}t}    & = -k_{5} [\text{NADH}] [\text{H}] + k_{6} [\text{O}_2]\frac{ [\text{H}_2\text{O}]}{K_m + [\text{H}_2\text{O}]}
    \end{aligned}
$$

其中,$[\text{ATP}]$、$[\text{ADP}]$、$[\text{Pi}]$、$[\text{NADH}]$ 和 $[\text{H}]$ 分别表示三磷酸腺苷、二磷酸腺苷、无机磷酸盐、辅酶NADH 和 氢离子的浓度;$k_{1-6}$ 表示各反应的速率常数;$K_m$ 是酶促反应的 Michaelis 常数。

\subsection{模型中各物质初始浓度}
模型中各物质初始浓度如下表:


\begin{table}[H]
    \centering
    \begin{tabular}{ccc}
        \toprule%第一道横线
        物质 & 浓度 & 单位\\
        \midrule%第二道横线 
        ATP & 2.0 & mM\\
        ADP & 0.5 & mM\\
        Pi & 1.0 & mM\\
        NADH & 0.01 & mM\\
        H & 0.001 & mM\\
        $\mathrm{FADH_2}$ & 0.01 & mM\\
        $\mathrm{O_2}$ & 0.01 & mM\\
        \bottomrule
    \end{tabular}
    \caption{\textbf{模型中各物质初始浓度}}
\end{table}





\end{document}