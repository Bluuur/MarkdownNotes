\documentclass{ctexart}
% 设置页边距
\usepackage{geometry}
\geometry{left=3.18cm,right=3.18cm,top=2.54cm,bottom=2.54cm}

% 设置目录显示的深度
\setcounter{tocdepth}{3}

% 设置标号深度
\setcounter{secnumdepth}{3}

\title{抗乳腺癌 mRNA 疫苗设计}
\author{生信 2001 张子栋 2020317210101}
\date{\today}

% 正文区
\begin{document}


    \maketitle
    \thispagestyle{empty}

    \newpage
    \tableofcontents
    % 目录设为页码 0(从正文起为第一页)
    \setcounter{page}{0}
    % 目录的页脚设为空
    \thispagestyle{empty}

    \newpage


    \section{乳腺癌及其疫苗概述}

    \subsection{乳腺癌概述}

    乳腺癌是一种常见的女性恶性肿瘤\textsuperscript{\cite{ref1}},其发生与多种因素有关,如遗传、激素、免疫等。目前,针对乳腺癌的治疗手段主要包括手术、放疗、化疗、内分泌治疗和靶向治疗等。然而,这些治疗方法都存在一定的局限性和副作用,因此,开发新的治疗策略是迫切需要的。
    
    \subsection{抗乳腺癌疫苗研发现状}
    抗乳腺癌疫苗主要分为两类:预定义抗原和匿名抗原\textsuperscript{\cite{ref2}}。在预定义抗原类别中,GP2 疫苗是一种针对 HER2 阳性乳腺癌的疫苗,已经进入三期临床试验;此外同样针对 HER2 阳性乳腺癌的疫苗 AE37,也正在进行临床试验。在匿名抗原类别中,有一种疫苗叫做 Nelipepimut-S,它是一种针对 HER2 低表达乳腺癌的疫苗,已经进入了三期临床实验。 



    \section{乳腺癌 mRNA 疫苗设计}
    \subsection{乳腺癌相关抗原}
    首先需要选择一个合适的乳腺癌相关抗原作为 mRNA 疫苗的编码序列\textsuperscript{\cite{ref3}}。乳腺癌中常见的抗原有 HER2、MUC1、EGFR、PRLR、CA153 等,其中 HER2 是最具代表性和临床意义的靶点,因为它在约 20\%-30\% 的乳腺癌中过表达或扩增,与肿瘤的侵袭性、复发转移和预后密切相关。因此选择 HER2 作为 mRNA 疫苗的主要抗原。

    \subsection{mRNA 疫苗的序列结构及修饰方式的优化}
    优化 mRNA 疫苗的序列结构和修饰方式可以提高其稳定性、翻译效率和免疫原性。mRNA 疫苗的序列结构一般包括 5' 帽子、5' 非翻译区(UTR)、开放阅读框(ORF)、3'~UTR 和 Poly A 尾巴\textsuperscript{\cite{ref4}}。

    5' 帽子可以保护 mRNA 免受核酸酶的降解,同时促进 mRNA 与核糖体的结合和起始转录。一般使用 7-甲基鸟苷三磷酸($\mathrm{m^{7}Gppp}$)作为 5' 帽子。5'~UTR 和 3'~UTR 可以调节 mRNA 的翻译、半衰期和亚细胞定位。一般选择高表达基因或经过特殊优化的 UTR\textsuperscript{\cite{ref4}}。例如,可以使用含有内部核糖体进入位点(IRES)或自我扩增序列(SAM)的 UTR 来增强 mRNA 的表达。ORF 是编码抗原蛋白的序列,一般需要进行密码子优化,即将很少使用的密码子替换为编码相同氨基酸残基的更经常使用的密码子,以提高 mRNA 在宿主细胞中的适应性和转录效率\textsuperscript{\cite{ref4}}。Poly A 尾巴可以延长 mRNA 的半衰期,并与 5' 帽子协同作用促进转录起始。一般需要足够长(100-150~bp)的 Poly A 尾巴才能发挥最佳效果。除了上述结构元件外,还需要对 mRNA 进行修饰核苷酸替换,以避免被免疫系统识别为外源性或病毒性 RNA 而引发干扰素反应或细胞凋亡。常用的修饰核苷酸有 1-甲基假尿嘧啶($\mathrm{m}1\Psi$)和 5-甲氧基胞嘧啶(mo5C)\textsuperscript{\cite{ref5}}。

    \subsection{递送系统}
    选择一个合适的递送系统,以保证 mRNA 能够有效地进入靶细胞并释放出来。目前最常用的递送系统是脂质纳米颗粒(LNP),它可以将 mRNA 包裹在一层双分子层的脂质中,形成类似于细胞膜的结构,从而保护 mRNA 免受降解,同时促进 mRNA 与细胞膜的融合和内吞作用。

    鉴于作业对递送系统没有做强制要求,所以这里不对递送系统做具体设计。
    
    \subsection{乳腺癌的 mRNA 疫苗具体结构及作用原理}
    综合前文所述,本文设计的抗乳腺癌 mRNA 疫苗如下:
    \begin{enumerate}
        \item 5'帽子:使用 $\mathrm{m^{7}Gppp}$ 作为 5'帽子;
        \item 5'~UTR:使用含有 SAM 序列的 5'~UTR,以实现 mRNA 在细胞内的自我扩增;
        \item ORF:使用编码 HER2 全长或部分片段(如 ECD 或 ICD)的序列,并进行密码子优化;
        \item 3'~UTR:使用高表达基因或经过特殊优化的 3'~UTR;
        \item Poly A 尾巴:使用长度为 100~bp 左右的 Poly A 尾巴;
        \item 修饰核苷酸:使用 $\mathrm{m}1\Psi$ 和 mo5C 替换部分或全部 U 和 C;
    \end{enumerate}    

    这种 mRNA 疫苗的作用机理是:通过注射或其他途径将 mRNA 疫苗进入人体后,LNP 可以保护 mRNA 免受降解,并促进其与靶细胞的结合和内吞。在细胞内,mRNA 可以通过 SAM 序列实现自我扩增,并借助 5’帽子和 Poly A 尾巴与核糖体结合进行转录,产生 HER2 抗原蛋白。HER2 抗原蛋白可以通过 MHC I 类或 MHC II 类分子呈递给 T 细胞,从而激活特异性 CD8+ 细胞毒性 T 淋巴细胞(CTL)和 CD4+ 辅助 T 淋巴细胞(Th)。CTL 可以识别并杀死表达 HER2 的乳腺癌细胞,Th 可以分泌细胞因子并协助 B 淋巴细胞产生针对 HER2 的抗体,从而形成体液免疫应答。这样,mRNA 疫苗就可以诱导出针对乳腺癌的细胞免疫和体液免疫,从而达到预防或治疗的目的。


    \begin{thebibliography}{99}
        \bibitem{ref1} 世界卫生组织。乳腺癌。2023-11-14. https://www.who.int/zh/news-room/fact-sheets/detail/breast-cancer 
        \bibitem{ref2} Lin MJ, Svensson-Arvelund J, Lubitz GS, Marabelle A, Melero I, Brown BD, Brody JD. Cancer vaccines: the next immunotherapy frontier. Nature Cancer. 2022 Aug 23;3(8):911-926. doi: 10.1038/s43018-022-00418-6. PMID: 35999309.
        \bibitem{ref3} Pardi N, Hogan MJ, Porter FW, Weissman D. mRNA vaccines — a new era in vaccinology. Nat Rev Drug Discov. 2018;17(4):261-279. doi:10.1038/nrd.2017.243
        \bibitem{ref4} Karikó K, Buckstein M, Ni H, Weissman D. Suppression of RNA recognition by Toll-like receptors: the impact of nucleoside modification and the evolutionary origin of RNA. Immunity. 2005;23(2):165-175. doi:10.1016/j.immuni.2005.06.008
        \bibitem{ref5} lamon DJ, Clark GM, Wong SG, Levin WJ, Ullrich A, McGuire WL. Human breast cancer: correlation of relapse and survival with amplification of the HER-2/neu oncogene. Science. 1987;235(4785):177-182. doi:10.1126/science.3798106

    \end{thebibliography}
\end{document}